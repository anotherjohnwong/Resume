\documentclass[centered,11pt,line]{res}

\usepackage[	bookmarks=true,
			linktoc=section,
			pdftitle={John Wong's resume},
			pdfauthor={John Wong},
			pdfsubject={Technical resume},
			pdfkeywords= {Xcode} {Instruments} {Git} {subversion} {C} {Objective-C} {Fortran} {Java} {Javascript} {PHP} {SQL} {basic shell scripting} {OpenCL} {MPI} {OpenMP} {IDL} {Matlab} {Mathematica} {LaTeX}{Python}
		]{hyperref}
		
\usepackage[bottom=0in,left=.7in,right=1in,top=.3in]{geometry}

\textwidth=7in
\parsep=0in

\newcommand{\tabulated}{\begin{tabular}{@{}p{5.5in}p{0.8in}}}
\newcommand{\block}{\begin{tabular}{@{}p{6.0in}}}
\renewcommand{\namefont}{\bf\LARGE}
\renewcommand{\sectionfont}{\bf\Large}
\newcommand{\shrink}{\vspace{-.15in}}

\begin{document}
	\name{\LARGE John Wong}
	\address{\href{mailto:anotherJohnWong@gmail.com}{anotherJohnWong@gmail.com} $|$ \href{skype:j.hnw.ng?userinfo}{Skype: j.hnw.ng}  $|$ \href{http://linkedIn.com/in/anotherJohn}{LinkedIn: anotherJohn}}
	\pagestyle{empty}
	\begin{resume}
		\fullline\vspace{-.2in}
		\section{\sc Summary of Qualifications}\vspace{.12in}
		\begin{description}
		\setlength{\itemsep}{-.02in}
			\item[Broad] interdisciplinary computational, programming, coding, and data analytic experiences.
			\item[Demonstrated] ability in rapidly adopting unfamiliar languages, frameworks, and methodologies.
			\item[Exposure] to industry technologies and practices such as noSQL solutions and agile development.
			\item[Proven] track record in tearing down existing code for debugging and performance tune-up.
		\end{description}\shrink
		
		%%%%%% Education %%%%%%
		\section{\sc Education}
		\tabulated
			{\bf University of Colorado at Boulder} \newline
			Ph.D. (projected Aug'13), M.S. Atmospheric and Oceanic Sciences & 2008 -- 2013 \\ \\
			{\bf University of Arkansas, Fayetteville}\newline
			M.A., B.S. Physics (Computational); B.S. {\it magna cum laude}  Mathematics (Applied) & 2003 -- 2007
		\end{tabular}\shrink
		
		%%%%%% Computer Skills %%%%%%
		\section{\sc  Technical skills}\vspace{.12in}
		\begin{description}
		\setlength{\itemsep}{-.02in}
			\item[Languages:] C/C++, Java, Python, Objective-C, Fortran, Javascript, PHP, SQL
			\item[Frameworks and libraries:]  OpenCL, MPI, OpenMP, Prototype, Dojo Toolkit
			\item[IDEs and tools:] vi, Xcode, Instruments, Git, subversion
			\item[Data formats:] XML, JSON, NetCDF, HDF5, GTFS
			\item[Other tools:] IDL, Matlab, Mathematica, \LaTeX, basic *NIX scripting
		\end{description}\shrink
		
		%%%%%% Projects %%%%%%
		\section{\sc Selected Projects}
		\tabulated
			\href{http://www.nrcm.ucar.edu}{\bf Nested Regional Climate Model} \newline
			Assisting in the development of a next-generation climate model. & 2012 \\ \\
			\href{http://www.geosci-model-dev.net/6/429/2013/}{\bf Lightning parameterization at the convective scale} \newline
			Implementing scale-aware lightning parameterization for weather models. & 2010 \\ \\
			{\bf Chemical kinetics with OpenCL} (class project) \newline
			Implemented a Rosenbrock chemistry model with OpenCL across architectures. & 2010 \\ \\
			\href{www.atmos-chem-phys.net/12/11003/2012/}{\bf Transport of chemicals assessed with models and satellite observation} \newline
			A collaboration between scientists from NCAR, CU, NOAA, \& NASA JPL. & 2008 \\ \\
			{\bf Improvement to Matlab code for DNA data analysis} (hired position) \newline
			Vectorized and debugged Matlab codes for processing digital signals. & 2007 \\ \\
			{\bf Web-based application for generating ``concept inventory''}\newline
			Built from the ground up a website for hosting, generating, \& managing assignments. & 2006
		\end{tabular}\shrink
		
		%%%%%% Sourcecode contribution %%%%%%
		\section{\sc Sourcecode contributions}
		
		\tabulated
			{\bf Refactoring of lightning NOx driver} --- \href{http://www.mmm.ucar.edu/wrf/users/}{\it NCAR's WRF-Chem v3.5} \newline
			Refactoring old implementation and mediating collaborated contributions. & 2012 \\ \\
			{\bf Lightning NOx emission parameterization} --- \href{http://www.mmm.ucar.edu/wrf/users/}{\it NCAR's WRF-Chem v3.4} \newline
			Implemented lightning NOx emission option for convective-scale simulations. & 2011 \\ \\
			{\bf Online tendency diagnostics} --- \href{http://www.mmm.ucar.edu/wrf/users/}{\it NCAR's WRF-Chem v3.2} \newline
			Developed module for decoupling tendency diagnostics for chemical species. & 2009
		\end{tabular}\shrink
		
	\end{resume} \vspace{.3in}
	\fullline
	\hspace{-.5in}Full C.V. and resume available through \url{https://github.com/anotherjohnwong/Resume}
\end{document}