%
% John Wong
% Curriculum Vitae
%
\documentclass[overlap,12pt,centered]{res}

\usepackage[	bookmarks=true,
			linktoc=section,
			pdftitle={John Wong's CV},
			pdfauthor={John Wong},
			pdfsubject={Curriculum Vitae},
		]{hyperref}
\usepackage[us,12hr]{datetime}
\usepackage{fancyheadings}
\usepackage{lastpage}

\oddsidemargin -.5in
\evensidemargin -.5in
\textwidth=6.5in
\itemsep=0in
\parsep=0in

\newcommand{\tabulated}{\begin{tabular}{@{}p{1.4in}p{4.9in}}}
\newcommand{\block}{\begin{tabular}{@{}p{6.5in}}}
\renewcommand{\namefont}{\bf\Large}
\renewcommand{\sectionfont}{\bf\large}

\begin{document}
	\name{\LARGE John Wong\vspace*{.05in}}
	\address{\large Curriculum Vitae}
	
	\thispagestyle{empty}
	\pagestyle{fancyplain}
	\renewcommand{\headrule}{}
	\renewcommand{\footrule}{\hline}
	\lfoot{\href{http://www.github.com/anotherjohnwong/resume}{J. Wong's CV}}
	\cfoot{Page {\thepage} of 5}
	\rfoot{Last edit: {\mmddyyyydate\today}  {\currenttime}}
	
	\begin{resume}
		\tabulated
			{\bf Address:}	&	216 UCB, Boulder, CO 80309-0216 \\
			{\bf Email:}	&	\href{mailto:anotherJohnWong@gmail.com}{anotherJohnWong@gmail.com}\\
			{\bf Skype:}	&	\href{skype:j.hnw.ng?userinfo}{j.hnw.ng} \\
			{\bf Other links:} &	\href{http://www.linkedin.com/in/anotherjohn}{LinkedIn},
							\href{http://github.com/anotherjohnwong/resume}{Github} \\
		\end{tabular}
	
	
		%%%%%% Education %%%%%%
		\section{\sc Education}
		\tabulated
			2010 -- Aug '13 \newline (projected)
				&  {\bf Ph.D., Atmospheric and Oceanic Sciences} \newline
						\href{http://atoc.colorado.edu}{University of Colorado at Boulder} \newline
						\href{https://www2.ucar.edu/news/experts/mary-barth}{Advisor: Dr. Mary Barth (NCAR-ACD)} \newline
						{\it Dissertation: Budgeting Summertime Upper Tropospheric Ozone Enhancement}
		\end{tabular}
		
		\tabulated
			2008-- 2010	&  {\bf M.S., Atmospheric and Oceanic Sciences} \newline
						\href{http://atoc.colorado.edu}{University of Colorado at Boulder} \newline
						{Advisor: Dr. David Noone}
		\end{tabular}
		
		\tabulated
			2006 -- 2007	& {\bf M.A., Physics} \newline
						University of Arkansas, Fayetteville \newline
						{Advisor: Dr. John Stewart} \newline
						{\it Masters Thesis: Web-based Application for Automated Generation of Physics Concept Inventory}
		\end{tabular}
		
		\tabulated						
			2003 -- 2006	& {\bf B.S. {\it magna cum laude}, Physics} (Computational) \newline
						University of Arkansas, Fayetteville \newline
						{Advisor: Dr. Jiali Li} \newline
						{\it Thesis: DNA Detection with a Nanopore Device}
		\end{tabular}
		
		\tabulated						
			2003 -- 2006	& {\bf B.S. {\it magna cum laude},  Mathematics} (Applied) \newline
						University of Arkansas, Fayetteville \newline
						{\it Thesis: Chromatic Polynomial of Torus Networks}
		\end{tabular}
		
		\tabulated						
			2003 -- 2006	&  (minor) {\bf Computer Sci and Computer Engineering} \newline
						University of Arkansas, Fayetteville
		\end{tabular}

		%%%%%% Projects %%%%%%
		\section{\sc Research Experiences}		
		\tabulated
			2012 -- Present	&	\href{http://www.nrcm.ucar.edu}{\bf Nested Regional Climate Model (NRCM)}
				
				Assisting in a project at the National Center for Atmospheric Research (NCAR) to test and develop the regional chemistry module for a next-generation climate model across scales as well as utilizing climatological simulations to evaluate future pollution scenarios.
		\end{tabular}
		
		\tabulated
			2010 -- 2012	&	\href{http://www.geosci-model-dev.net/6/429/2013/gmd-6-429-2013.html}{\bf Lightning parameterization at the convective scale}
				
				As part of my ongoing research work with budgeting upper tropospheric summertime ozone enhancement, I have implemented a lightning parameterization module for WRF-Cem that is suitable for models running at resolutions that are transitional between fully-resolved and fully-parameterized convection.
		\end{tabular}
		
		\tabulated
			2010			&	{\bf Chemical kinetics with OpenCL}
				
				For the class project of High Performance Scientific Computing at the University of Colorado at Boulder, I produced a version of the Regional Acid Deposition Model version 2 with Rosenbrock integration method using OpenCL. The same (identical) kernel has been tested and successfully ran on  various CPUs and GPUs on platforms running Mac OS X 10.6.
		\end{tabular}

		\tabulated				
			2008 -- Present	&	{\bf Convective-scale transport of trace gases assessed with models and satellite observations}
				
				A collaboration between multiple scientists from NCAR, CU-Boulder, NOAA, and NASA JPL to quantify the contribution of North American summer-time convective transport to the distribution of ozone and carbon monoxide in the upper troposphere using both regional atmospheric chemistry models and satellite observations.
		\end{tabular}
		
		\tabulated				
			2007 -- 2008		&	{\bf Technical assistant} at Univ. of Arkansas
		
				Debugged and optimized existing Matlab programs for analyzing signals from solid state nanopore device.
		\end{tabular}
		
		
		%%%%%% Publications %%%%%%
		\section{\sc Publications}
		\href{http://www.geosci-model-dev.net/6/429/2013/gmd-6-429-2013.html}{\textbf{Wong, J.}, M. C. Barth, and D. Noone (2013): Evaluating a lightning parameterization based on cloud-top height for mesoscale numerical model simulations, Geosci. Model Dev., 6, 429-443, doi:10.5194/gmd-6-429-2013, 2013}
		
		\href{http://www.atmos-chem-phys.net/13/1607/2013/acp-13-1607-2013.html}{Noone. D., C. Risi, A. Bailey, M. Berkelhammer, D. P. Brown, N. Buenning, S. Gregory, J. Nusbaume, D. Schneider, J. Sykes, B. Vanderwende, \textbf{J. Wong}, Y. Meiller, and D. Wolfe (2013). Determining water sources in the boundary layer from tall tower profiles of water vapor and surface water isotope ratios after a snowstorm in Colorado. Atmos. Chem. Phys., 13, 1607--1623, doi:10.5194/acp-13-1607-2013.}
		
		\href{http://www.atmos-chem-phys.net/12/11003/2012/acp-12-11003-2012.html}{Barth., M.C. , J. Lee, A. Hodzic, G. Pfister, W. C. Skamarock, J. Worden, \textbf{J. Wong}, and D. Noone (2012). Thunderstorms and upper tropospheric chemistry during the early stages of the 2006 North American Monsoon. Atmos. Chem. Phys., 12, 11003-11026, doi:10.5194/acp-12-11003-2012.}
		
		%%%%%% Selected Oral Presentations %%%%%%
		\section{\sc Selected Oral Presentations}
		
		\textbf{Wong, J.}, M. Barth, and D. Noone.  Lightning NOx parameterization in WRF-Chem with emphasis on validation.  Invited talk at WRF-Chem Group Meeting, August 23, 2012; Boulder, CO.
		
		\textbf{Wong, J}. From gaming to scientific computing: An introduction to General Purpose programming with GPUs (GPGPU). Presentation at Department of Atmospheric and Oceanic Science student forum, February 16, 2011; Boulder, CO.
		
		\textbf{Wong, J.}, D. Noone, M. C. Barth, W. Skamarock, G. Grell, and J. Worden. Budget and structural properties of the UTLS ozone enhancement during North American monsoon. Invited talk at WRF-Chem Group Meeting, October 27, 2010; Boulder, CO.
		
		
		%%%%%% Selected Poster Presentations %%%%%%
		\section{ \sc Selected Poster Presentations}
		
		Bela, M., M. Barth, \textbf{J. Wong}, O. Toon, H. Morrison, M. Weisman, K. Manning, G. Romine, W. Wang, K. Cummings, K. Pickering, and the DC3 Science Team. (2013) Evaluation of Wet Scavenging for the May 29, 2012 DC3 Severe Storm Case. 14th Annual WRF Workshop; 2013 Jun 24 -- 29; Boulder, CO. (Abstract submitted)
		
		\textbf{Wong, J.}, M. Barth, and D. Noone. (2012) Parameterizing Lightning-Generated NOx at resolutions with Convective Parameterization for Upper Tropospheric Ozone Simulations. 12th Annual WRF Users' Workshop; 2012 Jun 26 -- 29; Boulder, CO.
		
		\textbf{Wong, J.}, M. Barth, and D. Noone. (2011) Lightning NOx Parameterization for Synoptic Meteorological-scale Predictions with Convective Parameterization in WRF-Chem. American Geophysical Union Fall meeting; 2011 Dec 5--9; San Francisco, CA.
		
		Noone, D., C. Risi, A. Bailey, D. Brown, N. Buenning, S. Gregory, J. Nusbaumer, J. Sykes, D. Schneider, B. Vanderwende, \textbf{J. Wong}, D. Wolfe. (2010) Atmosphere-surface water exchanges from measurements of isotopic composition at a tall tower in Boulder. American Geophysical Union Fall Meeting; 2010 Dec 13--17; San Francisco, CA.
		
		\textbf{Wong, J.}, D. Noone, M. C. Barth, W. Skamarock, G. Grell, and J. Worden. (2009) A budget of the summertime ozone anomaly of 2006 above southern United States using WRF-Chem. American Geophysical Union Fall Meeting; 2009 Dec 14--18; San Francisco, CA.
		
		\textbf{Wong, J.}, D. Noone, M. C. Barth, W. Skamarock, G. Grell, and J. Worden. (2008) Coarse-scale convective transport of CO and O$_3$ over 36 hours above southern United States. American Geophysical Union Fall Meeting; 2008 Dec 15--19; San Francisco, CA.
		
		
		%%%%%% Sourcecode contribution %%%%%%
		\newpage\section{\sc Sourcecode contributions}
		
		\block
		{\bf Lightning NOx driver} \\
		in \href{http://www.mmm.ucar.edu/wrf/users/}{\it WRF-Chem v3.5} \\
		Refactored old implementation of lightning nitrous oxides (NOx) emission module of WRF-Chem into two separate modules, each separately handle flash rate prediction and NOx emission respectively. Also mediate concurrent contribution from scientists from Florida State University.
		\end{tabular}
		
		\block
		{\bf Lightning-generated NOx for convective parameterized models} \\
		in \href{http://www.mmm.ucar.edu/wrf/users/}{\it WRF-Chem v3.4} \\
		Implemented lightning NOx emission option into WRF-Chem for convective parameterized scale simulations based on Price and Rind (J. Geophys. Res., 1992) parameterization and Ott et al (J. Geophys. Res., 2010) emission guidelines.
		\end{tabular}
		
		\block
		{\bf Online tendency diagnostics} \\
		in \href{http://www.mmm.ucar.edu/wrf/users/}{\it WRF-Chem v3.2} \\
		Developed module for decoupling tendency diagnostics for chemical species and producing accumulated diagnostic outputs.
		\end{tabular}
		
		%%%%%% Computer Skills %%%%%%
		\section{\sc Technical Skills}\vspace{.15in}
		\begin{description}
		\setlength{\itemsep}{-.02in}
			\item[Languages:] C/C++, Java, Python, Objective-C, Fortran, Javascript, PHP, SQL
			\item[Frameworks and libraries:]  OpenCL, MPI, OpenMP, Prototype, Dojo Toolkit
			\item[IDEs and tools:] vi, Xcode, Instruments, Git, subversion
			\item[Data formats:] XML, JSON, NetCDF, HDF5, GTFS
			\item[Other tools:] IDL, Matlab, Mathematica, \LaTeX, basic *NIX scripting
		\end{description}
		
		%%%%%% Upperlevel Coursework %%%%%%
		\section{\sc Upperlevel Courseworks}
		
		{\bf Computer Science} \\
		Artificial Intelligence, Database Management Systems, Formal Languages and Computability, Graph and Combinatorial Algorithms, High Performance Scientific Computing
		
		{\bf Mathematics} \\
		Genetic Algorithms, Advanced Calculus, Numerical Analysis, Numerical Linear Algebra, Ordinary Differential Equations, Partial Differential Equations (PDE), Independent readings in Nonlinear PDE, Stochastic Processes
		
		{\bf Physics} \\
		Mathematical Methods in Electromagnetic Theory, Thermal Physics, Quantum Mechanics, Applied Group Theory in Physics, Fluid Instability \& Turbulence
		
		{\bf Atmospheric Science} \\
		Numerical Weather Prediction, Atmos. Chemistry, Atmospheric Dynamics (I \& II), Physical Oceanography, Radiative Transfer \& Remote Sensing, Clouds \& Aerosols
		
		\block\vspace{-.4in}
		%%%%%% Conference/Workshop Attendance %%%%%%
		\section{\sc Conference/Workshop Attendance}
		\tabulated
			3 -- 7 Dec, 2012	&	{ Amer. Geophys. Union Fall Meeting, San Francisco, CA } \\
			26 -- 29 Jun, 2012	&	{ 12th Annual WRF Users' Workshop, Boulder, CO  } \\
			5 -- 9 Dec, 2011	&	{ Amer. Geophys. Union Fall Meeting, San Francisco, CA } \\
			21 -- 25 Jun, 2010	&   	{ 11th Annual WRF Users' Workshop, Boulder, CO } \\
			16 -- 17 Jun, 2010	&  	{ TES Science Team Meeting, Pasadena, CA } \\
			14 -- 18 Dec, 2009	&	{ Amer. Geophys. Union Fall Meeting, San Francisco, CA } \\
			19 -- 22 Oct, 2009	&	{ Extra-Tropical UTLS Community Workshop, Boulder, CO } \\
			23 -- 26 Jun, 2009	&	{ 10th Annual WRF Users' Workshop, Boulder, CO }  \\
			23 -- 25 Feb, 2009	&	{ TES Science Team Meeting, Boulder, CO} \\
			15 -- 19 Dec, 2008	&   	{ Amer. Geophys. Union Fall Meeting, San Francisco, CA } \\
			23 -- 27 Jun, 2008	&	{ 9th Annual WRF Users' Workshop, Boulder, CO }  \\
			2005 -- 07 		&	{ Apple's WWDC 2005--2007, San Francisco, CA }
%			11 -- 15 Jun, 2007	&	{ Apple's WWDC 2007, San Francisco, CA } \\
%			07 -- 11 Aug, 2006	&	{ Apple's WWDC 2006, San Francisco, CA } \\%
%			06 -- 10 Jun, 2005	&	{ Apple's WWDC 2005, San Francisco, CA }
		\end{tabular}
		\end{tabular}

		%%%%%% Honors, Awards and Scholarships %%%%%%
		\section{\sc Honors, Awards \& Scholarships}
		\tabulated
			2012		&	{Department of Atmospheric and Oceanic Sciences Best Poster Award} \\
			2011		&	{United Government for Graduate Student Travel Grant} \\
			2005 -- 2007	&	{Apple's Worldwide Developer Conference Student Scholarship} \\
			2005 -- 2006	&	{Foundation of International Exchange Students Scholarship} \\
			2005 -- 2006	&	{Droke-Dunn Award for Outstanding Senior Math Major} \\
			2005 -- 2006	&	{Robert D. Maurer Research Scholarship for Physics Major}  \\
			2004 -- 2006	&	{David P Richardson Math Departmental Scholarship}  \\
			2004 -- 2006	&	{College of Engineering Scholarship} \\
			2004 -- 2005	&	{Univ. of Arkansas Chartwell's Room and Board Scholarship} \\
			2004 -- 2005	&	{Physics Departmental Scholarship} \\
			2004		&	{First Place in 2004 ACM Collegiate Programming Contest} \\
			2003 -- 2005	& 	{Engineering Dean's List}
		\end{tabular}
				
		%%%%%% Teaching Experience %%%%%%
		\section{\sc Teaching Experience}
		\tabulated
			01 -- 05/2013
				&	{\bf Teaching Assistant} for ATOC 1050 Weather and Atmos. \newline
					University of Colorado at Boulder
		\end{tabular}
		
		\tabulated
			2008 -- 2013
				&	{\bf Lab Instructor} for ATOC 1070 Weather and Atmos. Lab  \newline
					University of Colorado at Boulder
		\end{tabular}
		
		\tabulated
			2004 -- 2006
				&	{\bf Supplemental Instructions Leader} for Math and Physics \newline
					Enhanced Learning Center, University of Arkansas
		\end{tabular}
		
		%%%%%% Memberships and Affiliations %%%%%%
		%\section{\sc Memberships \& Affiliations}
		%\tabulated
		%	2008 -- Present	&	American Geophysical Union \\
	%		2005 -- 2010		& 	Apple Developer Connections \\
%			2007 -- Present	& 	Honor Society of Phi Kappa Phi \\
%			2005 -- Present	&	$\Theta$T Professional Engineering Fraternity Upsilon Chapter \\
%			2005 -- 2006		&	Macintosh User Group at the University of Arkansas \\
%			2005 -- 2006		&	Pi Mu Epsilon Undergraduate Math Club (Secretary) \\
%			2004 -- 2005		&	Diversity Alliance at the Univ. of Arkansas (VP/co-founder)
%		\end{tabular}
		
	\end{resume}
\end{document}